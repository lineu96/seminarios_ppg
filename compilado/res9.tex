% ----------------------------------------------------------------------
% Impacto econômico da COVID-19 no Brasil: podemos culpar o isolamento social? 
% ----------------------------------------------------------------------

Resenha do seminário ministrado no dia 19/06/2020 no Programa de Pós Graduação em Informática da UFPR por Luis Celso Gomes (professor do Departamento de Informática UTFPR).

No seminário ministrado pelo professor Luis Celso foram abordados aspectos econômicos da pandemina de Covid-19, uma discussão motivada pelo fato de que o Brasil vem se baseando na postura de países mais ricos que passaram pela pandemia. O problema desta abordagem é que as condições financeiras destes países permitem uma recuperação menos demorada, o que não é possível no Brasil. 

Como disse o professor, trata-se de uma crise que não é comparável com nenhuma outra da história pois nunca houve uma crise tão profunda, que afetasse o mundo por completo e ainda associada a um problema de saúde com fim incerto. 

Tomando os Estados Unidos como exemplo, trata-se do maior índice de desemprego causado por uma crise. O ponto é que se um país rico entra em recessão, um cidadão médio retorna à sua antiga condição financeira de forma mais rápida que o cidadão médio de um país em desenvolvimento, como o Brasil.

Mas até que ponto podemos culpar as medidas restritivas à condição econômica do país? Indicadores mostram que o Brasil é um dos países mais afetados pela pandemia e sofreu impacto econômico maior do que países do BRICS e sofre mais que a média mundial.

Porém vale ressaltar que antes do vírus chegar ao Brasil já podiam ser observados efeitos na economia causado pelo impacto na pandemia nos países desenvolvidos. 

Outro fator importante relacionado ao impacto da pandemia é a implementação de medidas menos pesadas de isolamento social. Quando comparados os índices de redução de poluição causada por automóveis, é possível notar que o Brasil possui uma das menores reduções no início da pandemia. Mostrando que quando o vírus chegou, poucas pessoas se mantiveram em isolamento.

Logicamente, a ausência de medidas mais efetivas para controle da doença foi refletida no cenário que vemos hoje, em que o Brasil se tornou o epicentro da doença no mundo e líder em número diário de óbitos.

Um resultado interessante é a relação entre redução de mobilidade e o avanço no número de óbitos que mostra que a medida que reduz-se a circulação de indivíduos reduz-se também o número de óbitos causados pela doença, mostrando que medidas de isolamento são altamente necessárias para preservar vidas.

Por fim, o professor Luis mostrou um estudo de caso a respeito do setor de aviação. Trata-se de um setor altamente rentável e que, mundialmente movimenta mais capital que países inteiros. Contudo foi um dos setores mais afetados com a pandemia a nível mundial devido à menor circulação de indivíduos, seja dentro do país ou para fora dele. O ponto central da discussão a respeito do impacto da aviação foi de que no começo da pandemia fazia sentido restringir a circulação de indivíduos a nível mundial com o intuito de impedir a disseminação, isto é, não permitir que pessoas contaminadas cheguem a diferentes lugares. Contudo, à medida que o globo é afetado por inteiro esta medida perde importância pois chega-se a um nível que a doença já se espalhou e não faz mais sentido reduzir o tráfego aéreo para fins de redução de contágio.


